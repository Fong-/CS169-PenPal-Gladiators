% Based on the documentation written by Bryan Huang

\documentclass[12pt]{article}
\usepackage[top=1in, bottom=1in, left=1in, right=1in]{geometry}
\usepackage{hyperref}
\setlength\parindent{0pt}

\title{PenPal Gladiators User Documentation}
\author{Version 1.0.1}

\begin{document}
\maketitle
\tableofcontents

\section{Welcome!}
What is Penpal Gladiators? It's a place where you can engage in open dialogue
with people of differing opinions. Our hope is that through Penpal Gladiators,
you'll be able to better understand those whose political views differ from
yours, and in the process learn something about yourself too. So, ready to
gladiate? Let's get started!

\section{Creating an Account}
Creating an account is super simple! There are three steps: registering an email
and password, selecting survey topics, and completing a survey.

\subsection{Registering}
If this is the first time you've visited the app, go ahead and enter your email
and password to register.  Don't worry about your username yet -- that'll come
later.

\subsection{Selecting Survey Topics}
Before we can match you with others, we need to get an idea of where you stand.
We'll ask you to select some topics that interest you, and ask you some
questions related to those topics.

\subsection{Completing the Survey}
When you're ready, go ahead and start answering the survey questions. You can go
back at any time to change topics or answers. When you're done, there will be a
summary page showing the topics and answers you've chosen. You can go back and
edit any answers now – if not, you're all set!

\section{Editing Your Profile}
It's important for other gladiators to get a sense of who you are, so we highly
recommend completing your profile. To get to your profile, just click ``Profile"
on the top right corner of the page.\\

Here, you can change your username, provide some info about your political
stance, or write a short summary about yourself. You can also change your email
and password as well.

\section{Finding a Match}

\subsection{Finding Other Gladiators}
Once your account and profile are all good to go, let's look for people to
gladiate with! Simply click ``Match" on the sidebar on the right side of the
page, and you'll get a list of gladiators who are good matches for you based on
the responses you provided in the survey. Pick a gladiator who you want to get
matched with, and a request will be sent to them.

\subsection{Pending Matches}
You can track the requests you've sent out to other gladiators here. It'll let
you know whether they've accepted or declined your request.

\subsection{Incoming Matches}
Here you can see requests from other gladiators to be matched with you. Feel
free to either accept or decline them.

\section{Gladiating}
Now that you have someone to gladiate with, let's start gladiating! Gladiating
is very similar to the online communication that you're used to -- it's a cross
between a chat conversation and a forum.

\subsection{Starting a Conversation}
To get started, let's start a conversation if there isn't one yet. Just click
``Start a new conversation", and pick a topic or question you'd like to gladiate
on.

\subsection{Post Types}
You can interact with the conversation through three ways -- posting a simple
reply, posting a summary, or working on your resolution. We'll get to resolution
later.

\subsubsection{Reply Posts}
This is the simplest kind of post, and it's just like any post you've ever made
on other sites. Here you can have a back-and-forth conversation with the other
gladiator.

\subsubsection{Summary Posts}
Throughout the conversation you may feel like you want to summarize what's been
said so far between you and the other gladiator, or offer your perspective of
what they've said, maybe for clarification. You can do this through the summary
post, and once it gets submitted they'll see it and either approve it, or not
approve it if they don't see eye to eye with you.  You'll also see summary posts
sent from the other gladiator, and you can choose whether to approve them or
not.

\subsubsection{Resolution Posts}
The last kind of post is called a resolution, and while it's not exactly a reply
post like the other two, it's still very important. Our goal is for you to be
able to build to a consensus on the conversation, whether that's coming to a
mutual understanding, or even agreeing to disagree.  Throughout the
conversation, both you and the other gladiator can contribute to the resolution
post with points that both of you agree on.

\subsection{Reaching a Consensus}
The last kind of post is called a resolution, and while it's not exactly a reply
post like the other two, it's still very important. Our goal is for you to be
able to build to a consensus on the conversation, whether that's coming to a
mutual understanding, or even agreeing to disagree.  Throughout the
conversation, both you and the other gladiator can contribute to the resolution
post with points that both of you agree on.

\end{document}
