% Based on the documentation written by Lucas yan

\documentclass[12pt]{article}
\usepackage[top=1in, bottom=1in, left=1in, right=1in]{geometry}
\usepackage{hyperref}
\usepackage{listings}
\setlength\parindent{0pt}

\title{PenPal Gladiators Technical Documentation}
\author{Version 1.0.1}

\begin{document}
\maketitle
\tableofcontents

\section{A Word About Licensing}
This app is licensed under GPLv3.  Anyone is free to modify this app so long as
they make the source code available under GPLv3 or later.  Modifying this app
and not making the source code available under GPLv3 is a violation of the
license for this app’s code and will result in legal action.

\section{Setting Up Your Development Environment}
\subsection{Git Setup}
Create an account on \href{https://github.com}{GitHub}and run
\begin{verbatim}
    git clone https://github.com/Fong-/CS169-PenPal-Gladiators.git
\end{verbatim}
to clone the repository.  See \href{https://git-scm.herokuapp.com/docs}{the
Git documentation} if you're not familiar with Git.


\subsubsection{PenPal Gladiators Git Conventions}
All features and changes are performed on a branch.  This is called “branch per
feature” development.  When a feature has been reviewed and is passing all tests
(e.g. via CI), rebase the branch on the iteration branch (if applicable) or
master branch (if not) and merge into the iteration branch or master by running:
\begin{verbatim}
git merge --no-ff [feature-branchname]
\end{verbatim}

\subsection{Heroku}
Download the \href{https://heroku.com}{Heroku command line tools} and add the Heroku remote:
\begin{verbatim}
heroku git:remote -a [projectname]
\end{verbatim}
inside the project directory. This adds the Heroku repository to your local
copy. Run
\begin{verbatim}
git push Heroku master
\end{verbatim}
to deploy the master branch to Heroku

\subsection{CodeClimate}
CodeClimate is linked to GitHub. Sign up for a
\href{https://codeclimate.com}{CodeClimate} account if you would like it to run
on your own repository. CodeClimate flags potential design issues like
duplication and excessive complexity before they turn into maintainability
problems.

\subsection{TravisCI}
TravisCI is an automated testing suite for our repo. Links to it can be found on
our GitHub repo.  Register for a \href{https://travis-ci.org/}{TravisCI} account
and link your GitHub repository. While CodeClimate checks for the style of the
code, TravisCI runs the app's test suite for correctness. Always look at the
TravisCI results before merging anything to master, and always write tests for
new features as they're implemented.

\section{Customizing Content and Options}
\subsection{Overview}
In order to add your own topics, survey questions, and response weights, you
will need to first erase the old content, modify \texttt{db/seeds.rb}, and then reseed.
We have provided a tool for you to erase old content without erasing user data
which you can run in the project directory with
\begin{verbatim}
rails runner tools/reset_seeds.rb
\end{verbatim}
or
\begin{verbatim}
heroku run rails runner reset_seeds.rb
\end{verbatim}
if on Heroku. You can also reseed by running
\begin{verbatim}
rake db:seed
\end{verbatim}
or
\begin{verbatim}
heroku run rake db:seed
\end{verbatim}
if on Heroku.

\subsection{Changing Topics}
\begin{enumerate}
    \item Open \texttt{db/seeds.rb} in a text editor of your choice
    \item Find the line that starts with \texttt{topics = [}
    \item You should see an indented set of lines in the format
    \begin{verbatim}
    { :name => "topic name", :icon => "topic icon path"},
    \end{verbatim}
    Each line represents a topic. Edit, add, or remove as necessary.
\end{enumerate}

\pagebreak
\subsection{Changing Questions of a Topic}
\begin{enumerate}
    \item Open \texttt{db/seeds.rb} in a text editor of your choice
    \item Create a new question and response set using the following template:
        \begin{lstlisting}[frame=single, numbers=left, breaklines=true]
name_of_question = Topic.find_by_name("Topic")
.survey_questions.create(:text => "Survey Question text", :index => Question Number)
responses = [
    {
    :index => Response Number,
    :text => "Response Text",
    :summary_text => "Summary Text"
    },
    ].map { |data| name_of_question.survey_responses.create data }
        \end{lstlisting}
        Each question and response must be assigned an index (the questions and
        responses are 0-indexed). For example, if this was the first question, the
        question index would be 0. If it was the second, it would be 1.
    \item To add the weights between responses, use this template for each pair
        of responses:
        \begin{lstlisting}[frame=single, numbers=left, breaklines=true]
ResponseWeight.create :response1_id => responses[Response Number].id, :response2_id => responses[Response Number].id, :weight => [weight]
        \end{lstlisting}
\end{enumerate}

\section{Customizing and Extending the Project}

\subsection{Database Models}
These reside in \texttt{app/models}:
\begin{itemize}
    \item \texttt{User} -- stores the information for each user.
    \item \texttt{Topic} -- represents a survey topic.
    \item \texttt{SurveyQuestion} -- belongs to a survey topic, represents a question in the topic
    \item \texttt{SurveyResponse} -- belongs to a survey question, represents a possible answer to a question
    \item \texttt{UserSurveyResponse} -- represent’s a user’s response to a question, Belongs to a SurveyResposne as well as a User
    \item \texttt{ResponseWeight} -- determines the score or ranking between two answers to the same question, a higher score will increase the likelihood of two users with those answers being matched.
    \item \texttt{Post} -- represents a message in a conversation
    \item \texttt{Arena} -- represents a match between two users
    \item \texttt{Conversation} -- represents a conversation on a topic between two users, belongs to an Arena
    \item \texttt{Invite} -- represents a match request from one user to another
\end{itemize}

Note: Nearly all models have their own controllers in \texttt{app/controllers}.
The controllers publicly expose model functions.

\section{Project Structure}
Our project uses LESS, HTML, CoffeeScript and Angular.

\subsection{File Locations}
\begin{itemize}
    \item Model (\texttt{.rb}) files are stored in \texttt{app/models}
    \item Controllers (\texttt{.rb}) files are stored in \texttt{app/controllers}
    \item Base level HTML pages are stored in \texttt{public/}.
        \texttt{homePage.html} is the main application view, \texttt{landingPage.html} is the
        landing page displayed to a first-time visitor, and
        \texttt{startPage.html} is the login and registration view
    \item Partial HTML fragments are stored in \texttt{app/assets/html}
    \item Front end CoffeeScript files are stored in \texttt{app/javascripts}
    \item Front end LESS files are stored in \texttt{app/stylesheets}
    \item Image resources for the topics and landing page are stored in \texttt{app/images}
    \item Fonts are stored in \texttt{vendor/fonts}
    \item Bootstrap CSS is stored in \texttt{vendor/stylesheets}
    \item Jasmine JS test code framework is stored in \texttt{vendor/javascripts}
\end{itemize}

\subsection{Writing and Running Tests}
Cucumber tests are stored in \texttt{features/}.  JavaScript tests are stored
in \texttt{spec/javascripts}. RSpec model tests are stored in
\texttt{spec/models}.  RSpec controller tests are stored in
\texttt{spec/controllers}.\\

Run JavaScript tests with:
\begin{verbatim}
rake spec:javascript
\end{verbatim}

Run RSpec tests with:
\begin{verbatim}
rspec
\end{verbatim}

Run Cucumber tests with:
\begin{verbatim}
cucumber
\end{verbatim}

\end{document}
